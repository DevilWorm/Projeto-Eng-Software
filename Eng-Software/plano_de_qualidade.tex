\chapter{Plano de qualidade}
\label{plano_de_qualidade}

\subsection{Importância da qualidade em projetos de Software}
É importante a qualidade em projetos de Software de maneira a minimizar riscos, quando o sistema é atualizado pode surgir erros nas funcionalidades do software ou a criar algo semelhante a uma função já existente. Além de evitar riscos como erros a qualidade em projetos de Software é importante para:\\
-haver uma boa compreensão do projeto\\
-Economizar recursos usados que irão constar no orçamento,\\
-Progresso na eficiência da equipa,\\
-Cumprir prazos,\\
-Melhor resultado no produto final,\\
-Satisfação do cliente.\\
A qualidade do projeto vem desde o início,os requisitos, é necessário defini-los e percebe-los juntamente com o cliente que pretendeu. Para ser possível manter essa qualidade é necessário haver um planeamento de controlo de qualidade de maneira a atender as normas do mercado e os requisitos. Os processos de gestão,controlo e planeamento deverão ser flexíveis de maneira a ser adaptado a cada software.\\
Ao implementar o plano de qualidade deverá estabelecer um padrão baseado nos requisitos e descrever como vai ser avaliado, para não deixar margem de haver suposições aos atributos por parte do engenheiro que devem ser otimizados.
\subsection{Responsabilidades do gestor da qualidade}
O gestor de qualidade tem um dos papeis mais importantes e é um dos responsáveis por manter a qualidade no desenvolvimento de um projeto de maneira a evitar erros e a seguir as normas. Ele é responsável por definir métodos de qualidade ,aprovar e analisar documentos de aprovação do software no seu estado final, atuar como suporte técnico da equipa de engenharia da empresa, realizar auditoria,coordenar reclamações vindas dos clientes sobre o produto que lhes diz respeito,implementar e administrar os requisitos das normas de gestão de qualidade.\\
Apesar das responsabilidades do gestor ,para alguém poder assumir é exigido pelo mercado experiências como já ter trabalhado com controle de qualidade, ser formado na área de engenharia e ter experiência na gestão de projetos, apesar de para ser um profissional na área é exigido mais competências.
\subsection{Padrões de referência}
Este capitulo foi construido com base na consulta e leitura dos seguintes sites:\\
-Site Linkedin,escrito por Claudia F. R. \\
https://pt.linkedin.com/pulse/importância-de-um-plano-qualidade-software-e-testes-farias-ctfl \\
-Site Project Builder\\
https://www.projectbuilder.com.br/blog/importancia-da-gestao-da-qualidade-em-projetos-2/
\subsection{Métricas}
Com o plano de qualidade pretende-se para o nosso sistema:\\
-Manter a qualidade do documento de requisitos;\\
-Estruturar a base de dados com qualidade e organização;\\
-Construir os Mock Up da ePadaria tendo em conta o necessário para ser intuitivo;\\
-Manter a organização e comunicação continua ao longo do desenvolvimento do projeto;\\