\chapter{Introdução}
\label{introdução}
\section{Descrição dos objetivos do documento}
\subsection{Enquadramento incial}
Este documento enquadra-se no desenvolvimento do projeto \textbf{ePadaria} , e tem como principais objetivos organizar, planear e solidificar um plano de projeto para o sistema.
A ideia do ePadaria começou pela cadeira de \textbf{Análise de Sistemas}, o tema foi um dos três temas escolhidos pelo nosso grupo e proposto pelo nossos professor, durante essa cadeira fomos aprendendo sobre requisitos e no final desenvolvemos o documento de requisitos para o ePadaria. No documento definimos os casos de uso, requisitos, os nossos atores, os pacotes do sistema e os nossos fluxos.\\
Continuamos o desenvolvimento do documento de requisitos, mas, com este documento desenvolvido para a cadeira de Engenharia de Software, onde iremos aprender mais um gênero de documento, outras práticas na escrita do mesmo e alargar os nossos conhecimentos.


\subsection{Público alvo}
O público alvo do nosso projeto são a maioria das padarias que se apresentam sem um sistema de gestão de pedidos.\\
O ePadaria só irá se destinar a pequenas padarias com a necessidade de passar do uso antiquado dos pedidos e notas em papel e caneta do uso mais atual, utilização de computadores e outros meios eletrónicos que nos permitem ajudar nas necessidades dessas pequenas padarias.

\section{O propósito do projeto a desenvolver}
O projeto \textbf{ePadaria} tem como objetivo apoiar os pequenos negócios, nomeadamente empresas pequenas, este sistema irá ajudar a simplificar a estrutura dos processos de uma padaria, tendo em conta o fornecimento, processamento, criação e entrega dos produtos.\\
As padarias locais são um estabelecimento que todos nós frequentamos mesmo que raramente, para além de ser um projeto de seguimento a uma cadeira anterior o facto de gostarmos dos produtos vendidos nas padarias e ser gerido muitas vezes por pessoas com quem relacionamos queremos ajudá-las com o nosso projeto. Podemos observar ao frequentar padarias que muitas não possuem sistemas informatizados e queremos mudar esse aspeto no mercado.
\\
