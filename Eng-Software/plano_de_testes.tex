\chapter{Plano de testes}
\label{plano_de_testes}
 A testagem tem como objetivo principal encontrar falhas no sistema realizando testes sistemáticos, mas para o projeto em geral serve para aumentar o nível de qualidade,  encontrando as falhas no sistema e as corrigindo. Faz parte da verificação e validação tal como em outras etapas do ciclo de desenvolvimento de software. 
 \section{Planeamento}
 Planear os testes necessários para criar uma aplicação segura e com um nível de usabilidade associado ao nosso publico alvo.\\
 Irão ser planeados rotas de testagem da aplicação para avaliar o nível de usabilidade, ataques ao software para garantir a segurança dos dados, entre outros.
 \section{Documentados}
 Estes testes serão documentados e avaliados, permitindo à equipa de desenvolvimento tomar escolhas e reavaliar a situação.\\
 A Documentação irá usar modelos universais, para permitir a leitura destes pelo a equipa e por futuros membros da equipa.
 \section{Executados}
 Depois de termos uma base solida de testes planeados e a documentação pronta, iremos executar os testes para garantir o sucesso da aplicação.
 \section{Repetidos}
 Estes testes poderão ser repetidos para garantir uma "sample pool" de testes, por exemplo com os relacionados com a usabilidade da aplicação, para avaliar se há outras falhas do design ou erros de programação a nível de aplicação\\
 \section{Conclusão}
Como já discutido, somos compostos apenas por uma equipa que está a acompanhar o desenvolvimento do software em todas etapas, logo naturalmente, também somos a equipa de testagem e vamos fazer todas etapas necessárias ditadas em cima para garantir uma aplicação segura e concretizada.