\chapter{Plano de segurança}
\label{plano_de_segurança}
\section{Segurança do nosso projeto}
A segurança no sistema ePadaria foi sempre algo que a nossa equipa pensou um bocado e com o tempo fomos aprendendo que é fundamental num sistema como este e a proteção dos dados pessoais do cliente é necessária para conseguirmos ter um programa que cumpra todas as leis e conformidades requeridas, mas também para que haja beneficíos seja para a equipa de desenvolvimento como para parceiros e clientes.\\

\subsection{Quais são os princípios da segurança de informação?}
Para conseguir resumir a complexidade da segurança de informação foram criados certos princípios a manter e são esses os mesmos que queremos praticar e aplicar no nosso projeto no seu desenvolvimento.\\
-Confidencialidade
A confidencialidade é um dos direitos dos nossos clientes e isso significa que a informação que eles inserem no nosso programa e que entra na base de dados, só pode ser obtida e atualizada por pessoas autorizadas e devidamente credenciadas. Informações e dados importantes dos nossos clientes nunca podem ser obtidos por terceiros ou desconhecidos.
Devem ser implementados mecanismos de segurança capazes de impedir que pessoas não autorizadas obetenham informações confidenciais.\\

-Confiabilidade
É a fidedignidade da informação. Deve ser assegurado ao utilizador, a boa qualidade dos dados com que ele estará a trabalhar.\\

-Integridade
Um dos príncipios mais importantes a considerar é que haja a garantia de que a informação estará exata, completa e preservada contra alterações sem autorização, fraude ou até mesmo contra a sua destruição.
Assim não só estaria comforme com o RGPD, mas também seriam evitadas violações de informação, sejam elas de propósito ou acidentais.\\

-Disponibilidade
A informação e dados devem estar continuamente disponíveis e acessíveis para pessoas autorizadas.\\

-Autenticidade
Tem como objetivo fazer e manter os registos apropriados, como quem realizou acessos a dados e quando, atualizações de dados incluíndo a eliminação dos mesmos, de modo que haja confirmação da sua autoria e originalidade.


\subsection{Testes de segurança}
Como referido no planeamento de testes, vão haver vários testes a ser realizados ao sistema para conseguir chegar ao nível desejado e requerido de segurança. Com estes mesmos vamos conseguir encontrar as falhas no nosso sistema e as corrigir, mas para isso vão ter que ser pensados, planeados, executados e repetidos.\\

\subsection{Realização de backups à base de dados e à aplicação}
Voltando aos princípios fundamentais, chegamos á conclusão que para manter esses mesmos, como por exemplo a disponibilidade dos dados, têm se que olhar para os cenários que ocorrem no mundo real. Devido á natureza do negócio vão ser armazenados um quantidade grande de dados a certo ponto e como responsáveis temos que preparar o sistema para falhas, quer seja um ataque cibernético ou uma simples falha. Qualquer um deles pode causar muito dano á empresa e aos seus clientes.\\
Para diminuir o dano causado ao sistema devem se realizar backups á base de dados, de forma continúa e organizada. Semana a semana, é o recomendado pela nossa equipa, mas poderá vir a diminuir no seu intervalo dependendo da quantidade de clientes que o sistema têm, mas o objetivo é conseguir manter uma cópia legítima dos dados de maneira a que se houver uma falha há uma recuperação mais rápida e mantém registos de informação legítima caso haja alteração não autorizada por terceiros.