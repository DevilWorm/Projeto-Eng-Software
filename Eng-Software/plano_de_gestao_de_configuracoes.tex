\chapter{Plano de gestão de configurações}
\label{plano_de_gestao_de_configuracoes}

\subsection{Enumeração de parâmetros e variáveis que podem alterar a forma como o sistema é usado ou apresentado}

\subsection{Descrição dos vários modos do sistema}
\subsubsection{Administradores}
\subsubsection{Utilizadores registados}
Para estar no modo registado,o utilizador primeiro terá de se registar no nosso Sistema, assim poderá ter acesso a sítios que não tinha antes. Os utilizadores registados irão poder ver o mesmo que no modo Público, depois de registados terão direito a aceder aos produtos disponibilizados, a que padaria estes pertencem com as respetivas informações de cada padaria e assim adicionar os produtos ao carrinho para os poder comprar.
\subsubsection{Público}
O modo público é semelhante ao modo de utilizadores registados. Quem estiver neste modo poderá visualizar a home page do ePadaria, os contactos e informação sobre o ePadaria como também as páginas de Log In e Sig in do cliente e o carrinho. Também poderá aceder ao log in do funcionário mas só em modo administrador é possível aceder á página de gestão de pedidos e produtos, como também para aceder aos produtos listados no ePadaria é necessário estar em modo Utilizadores registados.
\subsubsection{Administradores}
O modo Administradores é um modo restrito, esses utilizadores irão poder visualizar como público, mas não terão acesso à parte de comprar produtos. Em vez disso nesse modo terão acesso a uma página que só os administradores poderão aceder, essa página contém acesso para o administrador poder remover ou adicionar pedidos, visualizar os pedidos feitos que ainda aguardam conclusão, visualizar os pedidos já concluídos como também adicionar e eliminar produtos.
\subsubsection{Gestor do sistema}
O ePadaria não tem especificamente o modo gestor do sistema, o gestor acede diretamente ao código e aos dados através do Eclipse e do MySQL.