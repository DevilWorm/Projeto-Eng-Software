\chapter{Standards, técnicas e ferramentas}
\label{standards_tecnicas_ferramentas}

\section{Standards}
\subsection{Organização do código}
O código no que diz respeito á organização ao desenvolver o ePadaria irá ser organizado por parte do cliente e funcionário onde cada um irá ter acessos diferentes alem de aspetos em comum.
\subsection{Especificação}
O ePadaria é constituido pela parte do cliente e a parte do funcionário e ambos terão acesso ás páginas dos contactos e da informação sobre nós, sem haver necessidade de estar registado no ePadaria para ver essa informação.\\
O cliente vai poder se registar e efetuar o login no nosso sistema onde irá inserir os dados necessários para poder fazer pedidos pelo nosso sistema ás padarias pretendidas. Poderá tambem aceder aos produtos disponiveis e ao aceder a cada produto poderá ver a respetiva padaria ou ver os pedidos em apenas uma especifica padaria. \\
O funcionário vai ter uma parte no ePadaria que só ele poderá aceder onde irá conter o necessário para poder ver os pedidos efetuados pelos clientes, incluíndo os que já foram concluídos. Poderá ver as informações dos produtos pedidos pelo cliente, quem efectuou o pedido e a data e hora do pedido. Na parte dos produtos o funcionário pode eliminar ou acrescentar novos produtos para o cliente poder comprar ,terá de inserir a informação do produto referente ao nome e breve descrição.

\subsection{Documentação}
\subsection{Métricas}
\subsection{Comentários}

\section{Técnicas}

\section{Ferramentas}
Sendo que o ePadaria, que como sabemos irá nos permitir ajuda as pequenas padarias,ao qual o nosso sistema irá ser constituido pela parte do cliente que se trata da ponte de comunicação entre o lado Cliente e a Base de Dados,a parte da Padaria permite ver a informação dos pedidos, confirmar/rejeitar pedidos, adicionar pedidos, permite aceder à base de dados e uma base de dados para serem armazenados os dados e as receitas dos produtos da Padaria os que os mesmo achassem que não fosse confidencial,mas não seriam partilhadas com os clientes.\\
As ferramentas usadas para construir o ePadaria será programas que nos permitem criar bases de dados, a nossa interface como as suas funções. 
CONTINUAR,FALAR DOS PROGRAMAS? E QUE FERRAMENTAS MAIS?