\chapter{Standards, técnicas e ferramentas}
\label{standards_tecnicas_ferramentas}

\section{Standards}
\subsection{Organização do código}
O código no que diz respeito á organização ao desenvolver o ePadaria irá ser organizado por parte do cliente e funcionário onde cada um irá ter acessos diferentes alem de aspetos em comum.
\subsection{Especificação}
O ePadaria é constituido pela parte do cliente e a parte do funcionário e ambos terão acesso ás páginas dos contactos e da informação sobre nós, sem haver necessidade de estar registado no ePadaria para ver essa informação.\\
O cliente vai poder se registar e efetuar o login no nosso sistema onde irá inserir os dados necessários para poder fazer pedidos pelo nosso sistema ás padarias pretendidas. Poderá tambem aceder aos produtos disponiveis e ao aceder a cada produto poderá ver a respetiva padaria ou ver os pedidos em apenas uma especifica padaria. \\
O funcionário vai ter uma parte no ePadaria que só ele poderá aceder onde irá conter o necessário para poder ver os pedidos efetuados pelos clientes, incluíndo os que já foram concluídos. Poderá ver as informações dos produtos pedidos pelo cliente, quem efectuou o pedido e a data e hora do pedido. Na parte dos produtos o funcionário pode eliminar ou acrescentar novos produtos para o cliente poder comprar ,terá de inserir a informação do produto referente ao nome e breve descrição.

\subsection{Documentação}
Para implementar o ePadaria, tal como outros sistemas semelhantes, foi feito diversos tipos de documentos necessários,começando com o documento de requisitos. O documento de requisitos é a base do sistema que segue a seguir de aprovação.\\
Existem vários documentos usados á parte para o ePadaria que são dados como Plano. O plano de qualidade ajuda a manter a qualidade do sistema acompanhando desde o inicio ao fim do projeto, o plano de gestão de dados que fala sobre a base de dados e como os dados são geridos, o plano de gestão de recursos onde é mencionado recursos e como são afetados com o ePadaria,o plano de testes fala dos diversos tipos de testes realizados ao sistema,o plano de formação.


\section{Técnicas}
Ao longo do desenvolvimento da documentação e desenvolvimento do sistema ePadaria é aplicado técnicas dadas e aprendidas nas respetivas cadeiras de Analise e Desenho De Sistemas, Engenharia de Software e Programação para web.\\
Na cadeira de Analise e desenho de sistemas aprendemos técnicas e práticas para construir o documento de requisito, para além de onde surgiu a ideia do sistema ePadaria, que tem o intuito de ajudar as pequenas padarias sem sistema informático. É usado técnicas como construção de diagramas aos quais:\\ 
-diagramas de casos de uso;\\ 
-diagramas de classe;\\
-diagramas de objetos;\\
-diagramas de atividade\\
-diagramas de estados;\\
-diagramas de sequência;\\
-diagramas de colaboração;\\
-diagramas de componentes;\\
-diagramas de distribuição;\\
Alguns dos respetivos diagramas para além de usados na construção do documento de requisitos podemos visualizá-los ao longo deste documento. O documento de requisitos seguiu um modelo estruturado idealizado para o próprio documento, ao consultar outros documentos de requisitos podemos ver seguem o mesmo modelo ou um semelhante respeitando as normas da prática.\\
Em Programação para web as técnicas e práticas foram aplicadas no desenvolvimento do sistema, aprendemos o necessário para desenvolver o sistema com as práticas usadas em Web Apps. É usado a prática de javascript e java como base para construir o essencial do ePadaria, na cadeira é dado e aprendido o necessário para o implementar.\\
Em Engenharia de Software é aprendido técnicas para preparação de documentos seguindo um esboço de plano fornecido pelo professor e implementado no programa TextStudio. Apesar de novas técnicas aprendidas e workshops desenvolvidos entre eles workshop de diagrama de distribuição, mockups, organização do código e plano de testes, nesta cadeira implementamos técnicas aprendidas na de Analise e Desenho de Sistemas como criação de diagramas usados neste mesmo documento.\\ 


\section{Ferramentas}
Sendo que o ePadaria, que como sabemos irá nos permitir ajuda as pequenas padarias, ao qual o nosso sistema irá ser constituído pela parte do cliente que se trata da ponte de comunicação entre o lado Cliente e a Base de Dados, a parte da Padaria permite ver a informação dos pedidos, confirmar/rejeitar pedidos, adicionar pedidos, permite aceder à base de dados e uma base de dados para serem armazenados os dados e as receitas dos produtos da Padaria os que os mesmo achassem que não fosse confidencial, mas não seriam partilhadas com os clientes.\\
As ferramentas usadas para construir o ePadaria são programas que nos permitem criar bases de dados, a nossa interface como as suas funções. \\
O programa MySQL é um serviço de armazenamento de dados gerido para inserir aplicativos nativos na nuvem. O MySQL é o programa usado para guardar a base de dados do ePadaria, usando linguagem SQL. A base de dados constitui as diferentes tabelas para armazenar os dados dos clientes, produtos, padarias e dos funcionários.\\
Para construir o sistema ePadaria é usado o programa Eclipse IDE, o programa permite criar diversos tipos de projetos e ficheiros ao qual podendo usar vários tipos de linguagens de programação, de marcação e de consulta estruturada. No ePadaria é criado um projeto Dynamic Web Project com ficheiros JSP. As diversas linguagens usadas são:\\
- \textbf{HTML}  (Linguagem de Marcação);\\
- \textbf{SQL}  (Linguagem de consulta estruturada);\\
- \textbf{CSS}  (mecanismo para adicionar estilo a um documento web);\\
- \textbf{Java}  (Linguagem de Programação);\\
- \textbf{JavaScript} (Linguagem de Programação);\\
