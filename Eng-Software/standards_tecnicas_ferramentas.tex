\chapter{Standards, técnicas e ferramentas}
\label{standards_tecnicas_ferramentas}

\section{Standards}
\subsection{Organização do código}
\subsection{Especificação}
\subsection{Documentação}
\subsection{Métricas}
\subsection{Comentários}

\section{Técnicas}

\section{Ferramentas}
Sendo que o ePadaria, que como sabemos irá nos permitir ajuda as pequenas padarias,ao qual o nosso sistema irá ser constituido pela parte do cliente que se trata da ponte de comunicação entre o lado Cliente e a Base de Dados,a parte da Padaria permite ver a informação dos pedidos, confirmar/rejeitar pedidos, adicionar pedidos, permite aceder à base de dados e uma base de dados para serem armazenados os dados e as receitas dos produtos da Padaria os que os mesmo achassem que não fosse confidencial,mas não seriam partilhadas com os clientes.\\
As ferramentas usadas para construir o ePadaria será programas que nos permitem criar bases de dados, a nossa interface como as suas funções. 
CONTINUAR,FALAR DOS PROGRAMAS? E QUE FERRAMENTAS MAIS?