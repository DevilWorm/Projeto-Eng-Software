\chapter{Âmbito do Projeto}
\label{ambito}
\section{Apresentação}
\subsection{Contexto e Enquadramento}
Este sistema apresentou-se num contexto que muitos de nós nos encontramos na grande cidade do Porto e em todo Portugal a percentagem de pequenos negócios é maior que a percentagem de grandes empresas que existe no nosso país. Podemos ver que existem mais pequenas padarias por vezes mais que uma por rua o que se repara que parte são geridas pelos nossos vizinhos ou conhecidos. A maior parte dessas pequenas padarias já existe de negócios que foram por vezes passados de gerações anteriores ou negócios que existem á mais de vinte anos, e por essa razão essas pequenas padarias acabam por continuar sempre com o mesmo sistema de gestão de apontar tudo em papeis e a utilização de papel e caneta mesmo que saibam utilizar um computador. Com isso o nosso sistema pretende ajudar essas pequenas empresas/padarias.
A ePadaria enquadra-se nessa realidade, uma que facílmente pode ser preenchida por este mesmo sistema que só traria vantagens ao utilizador e ao consumidor no intúito de fornecer um sistema de facil utilização e manutenção para essas padarias. 
Com isto á que ter em atenção de que apenas se dirige a pequenas empresas de apenas um estabelecimento ,o que neste caso o ePadaria já não seria possível ser adquirido por empresas com mais que um estabelecimento
(((((Os principais problemas serião então a falta de um sistema de gerenciamento fácil de usar,intuitivo e barato para as padarias normais com a necessidade de tal.)))))
\section{Decomposição funcional do sistema}
A estrutura do ePadaria irá ser constituido pela parte do cliente que se trata da ponte de comunicação entre o lado Cliente e a Base de Dados,a parte da Padaria permite ver a informação dos pedidos, confirmar/rejeitar pedidos, adicionar pedidos, permite aceder à base de dados e uma base de dados para serem armazenados os dados.
\section{Constrangimentos do processo}
\subsection{Desempenho esperado}
Não se conhece o desempenho esperado por este sistema dado que não se conhece semelhantes e as as suas especificações, mas o desempenho desejado em geral é que seja fácil de usar com um rápido desempenho e de fácil uso e compreenção
\subsection{Limitações} 
As limitações encontradas são maior parte relacionadas com estabelecer contacto com as várias padarias,conseguir explicar bem as vantangens todas a possiveís clientes e intregar o nosso framework com o deles. Como tambem ter em conta que se destina a Padarias com apenas um local de venda ao público.