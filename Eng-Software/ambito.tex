\chapter{Âmbito do Projeto}
\label{ambito}
\section{Apresentação}
\subsection{Contexto e Enquadramento}
Este sistema apresentou-se num contexto que muitos de nós nos encontramos na grande cidade do Porto e em todo Portugal, existem mais padarias que ruas e maior parte são geridas pelos nossos vizinhos comuns de meia idade com pouco planeamento,conhecimento e pouco suporte :(.\\
A ePadaria enquadra-se nessa realidade, uma que facílmente pode ser preenchida por este mesmo sistema que só traria vantagens ao utilizador e ao consumidor.\\
Os principais problemas serião então a falta de um sistema de gerenciamento fácil de usar,intuitivo e barato para as padarias normais com a necessidade de tal.
\section{Decomposição funcional do sistema}
A estrutura do ePadaria irá ser constituido pela parte do cliente que se trata da ponte de comunicação entre o lado Cliente e a Base de Dados,a parte da Padaria permite ver a informação dos pedidos, confirmar/rejeitar pedidos, adicionar pedidos, permite aceder à base de dados e uma base de dados para serem armazenados os dados.
\section{Constrangimentos do processo}
\subsection{Desempenho esperado}
Não se conhece o desempenho esperado por este sistema dado que não se conhece semelhantes e as as suas especificações, mas o desempenho desejado em geral é que seja fácil de usar com um rápido desempenho e de fácil uso e compreenção
\subsection{Limitações} 
As limitações encontradas são maior parte relacionadas com estabelecer contacto com as várias padarias,conseguir explicar bem as vantangens todas a possiveís clientes e intregar o nosso framework com o deles. Como tambem ter em conta que se destina a Padarias com apenas um local de venda ao público.