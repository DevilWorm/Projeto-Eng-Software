\chapter{Descrição técnica dos sistemas}
\label{descricao_tecnica_dos_sistemas}

\section{Resumo da Especificação de requisitos}
Com a aplicação do sistema ePadaria vai se promover e otimizar a comunicação e funcionamento dos 3 setores, pedidos, cozinha e entregas.\\
Na gestão de pedidos, o gestor de pedidos, só terá que preencher dados na aplicação ePadaria, esta que trata de comunicar e verificar os stocks e também pedidos anteriores, que certifica que haverá espaço para o pedido no dia alocado, e dando ao gestor de pedidos imediatamente uma validação do pedido ou
refere o que estava invalido no pedido.\\
Na gestão de pedidos, o gestor de pedidos, só terá que preencher dados na aplicação ePadaria, esta que trata de comunicar e verificar os stocks e também pedidos anteriores, que certifica que haverá espaço para o pedido no dia alocado, e dando ao gestor de pedidos imediatamente uma validação do pedido ou refere o que estava invalido no pedido.\\

\section{Descrição do software que será necessário desenvolver}
\subsection{Requisitos funcionais}
\subsubsection{Casos de uso}
Na figura abaixo representa-se o modelo genérico de casos de uso do sistema ePadaria sob a forma de um diagrama de pacotes. Cada pacote agrega uma ou mais partes do sistema que se destinam a suportar processos da organização e/ou a reunir um conjunto de funcionalidades. Em cada pacote incluem-se alguns exemplos de Atores e casos de uso desenvolvidos para o sistema.

\subsubsection{Requisitos}
\section{Descrição do estrutura física do hardware}
Para se poder usar o site será mais viavél o seu uso num computador ou tablet que em um telemóvel devido á sua estrutura e ter sido destinado mais para uso no computador. 
Não é necessário se ter um dispositivo de alto desempenho para se poder aceder ao site e ser usado,apenas é necessário ter os requisitos minimos.