\chapter{Conclusão}
\label{conclusão}
\section{Reflexões}
O projeto foi realizado e desenvolvido por uma equipa com um nível de comunicação alto, e através dessa facilidade criou-se um sistema de desenvolvimento aberto a ideias e a mudanças, este processo abrir as portas a ideais novas, mas também conseguiu decidir quando algo estava errado ou aprofundado demasiado. Estas mudanças foram na maioria positivas, mesmo quando houve discussões sobre adicionar ou remover funcionalidades ou a mudança das mesmas, a equipa tomou sempre uma posição conjunta após o debate destas ideias.\\
A construção deste documento também criou um desenvolvimento mais pensado e realizado de funcionalidades que ainda não estavam propriamente pensadas, permitindo a criação e a separação de trabalho das mesmas entre a equipa.
\section{Áreas de Melhoramento}
No contexto de melhoramento do projeto, a equipa apercebeu-se de limites de conhecimento ou de tempo, que fechou vários caminhos para o projeto no plano atual, que no futuro poderão ser de novo abrangidas pela a equipa.\\
Estas áreas, como por exemplo o design da aplicação ou limitações de software, poderiam ser melhoradas ou atualizadas quando o projeto for retomado pela a equipa. 
\section{Espetativas de Trabalho Futuro}
Este projeto tem potencial para ser retomado pela a mesma equipa no futuro, durante o desenvolvimento do mesmo a equipa mostrou interesse em continuar a produção do mesmo, e criamos a aplicação já com a ideia que é algo que o mercado atual, com o crescimento de "take-aways" e "internet shopping" durante a pandemia Covid-19, poderá mostrar interesse, claro que a equipa ainda não realizou testes de mercado ou entrevistas de interesse mas sempre tivemos em mente o potencial de trabalho futuro na mesma.